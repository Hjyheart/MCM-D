%!TEX root = mcmpaper.tex
\begin{abstract}
In order to protect people's security and to provide more convenience, the reasonable setting of the airport security procedures is of great importance. This paper take the airport security inspection process as the main research object and use relevant theory, through the establishment and analysis of the model to find the bottlenecks in the current security inspection process and optimize the security system to get the best solution. And this paper also consider the influence of cultural differences and give some reasonable suggestions.\\
To establish the initial model, this paper first analyze the information given by the subject, and make reasonable assumptions. This paper use the queuing theory model to describe the queueing situation. And we transform the real airport security inspection system to a Petri net to simplify calculation. Then this paper use programming method to simulate the flow of passengers between various zones through computer simulation and give out the average passing time and its standard deviation on the chart, it's obvious that Zone A and B are the bottlenecks in the security process.\\
To optimizing the bottlenecks, this paper establish a evaluation index system including queuing satisfaction and cost effect. Firstly, we can increase the personnel and equipment in Zone A and B, but the values can not increase without limit considering the cost of modifications. Therefore, by this system, we can find out values that can not only greatly optimize the wait time but also save money. The result show that 11 officers in Zone A and 24 sets of equipment in Zone B will be the best program which can save over 50\% time. Besides, we can adjust the procedures, for the most crowded and time consuming Zone B, we can moderately increase the officer to sharply reduce the congestion of passengers and improve the travel experience.
\\
In order to find out the effect of different cultures, two culture differences are mentioned as analysis object. This paper quantified the influence and then change the corresponding parameter in the simulation program to get the result to make sensitivity analysis. And proposes are also given to accommodate these differences. For example, more equipment should be offered in places with more belongings.

\end{abstract}

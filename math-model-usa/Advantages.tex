%!TEX root = mcmpaper.tex
\section{Advantages and disadvantages and improvement}
\subsection{Advantages}
\begin{enumerate}
\item It is reasonable to assume that the hypothesis we set up before the model is based on the facts and the meaning of the study
\item The queuing theory we use to model the passenger's experience, which is difficult to measure directly, is represented by quantifiable indicators so that the passenger's experience can be compared with each other.
\item The data we get from simulation by computer has great scientific and reference value.
\item Through the image can be more obvious to see the appropriate solution.
\item The method we used can be applied to any airport.
\item The optimization recommendations we have given are practical and cost-effective, and are practical solutions.
\end{enumerate}

\subsection{Disadvantages}
\begin{enumerate}
\item The model established in this paper does not take into account the initial value of the number of queues in the simulation. We set the intial value 0. This will result in a certain impact. \textbf{The corresponding improvement scheme is to extend the period of simulation and get the steady state solution.}
\item The results obtained in this paper vary with the traffic volume, so the results can not be directly applied to other airports
\item We ignore the details of the security process for the impact of the results. For example, different staffs work differently and so on. \textbf{The corresponding improvement is that through a large zone of field investigation and data collection We can calculate the specific details of the results of our modeling and analysis of the specific impact of the quantified as a result of the variables into the model to improve the model.}
\end{enumerate}

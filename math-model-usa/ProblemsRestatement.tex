%!TEX root = mcmpaper.tex

%======================问题重述====================================
\section{Problems restatement}

\subsection{Introduction}
Following the terrorist attacks in the US on September 11,2001, airport security has been significantly enhanced throughout the world, the security situation has also been improved. In order to ensure the safety of all passengers, every passenger must pass the security checkpoint before entering the airport, but a lot of inconvenience is also caused. How to maximize security while minimizing inconvenience to passengers, the research of this problem is of great significance in the present air traffic business.

\subsection{Restatement}
The flow of US airport security checkpoints is known as followed: First line up to check the identification and board documents in Zone A, then check the body and belongings in Zone B, then collect belongings in Zone C, if there's nothing wrong, then you can leave. Passengers that fail this step receive a pat-down inspection in Zone D.
Besides, approximately 45\% of passengers enroll in a program called Pre-Check for trusted travelers, these passengers pay 85\$ to receive a background check and enjoy a separate screening process for five years, in this way they can save some time in Zone B. Data of time in each step has been offered.
\par
Our tasks are:
\begin{enumerate}
  \item Establish a reasonable model to identify bottlenecks.
  \item Develop two or more potential modifications to the current process to improve passenger throughput and reduce variance in wait time.
  \item Consider cultural differences and analyze how can the security system accommodate these differences.
  \item Propose policy and procedural recommendations for the security managers based on the model.
\end{enumerate}
